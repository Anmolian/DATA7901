\centerline{\Large\bfseries Abstract}
\bigskip

This research proposal aims to address the critical gap in temporal awareness within Large Language Models (LLMs), particularly in the context of the University of Queensland's development chatbot, ChatUQ. Despite the sophisticated capabilities of LLMs in information retrieval (IR) and response generation, their performance in processing time-sensitive queries remains markedly deficient, which can undermine their utility in dynamic and real-time environments. The overarching goal of this project is to enhance the temporal processing capabilities of ChatUQ by systematically identifying, developing, and integrating novel methodologies within its retrieval-augmented generation (RAG) framework. \\

To achieve this, the research will first quantify temporal awareness deficiencies through the creation of a robust dataset comprising diverse temporal queries. This dataset will facilitate the measurement and enhancement of the temporal accuracy of LLM-based IR systems. Furthermore, the project will review and adapt existing temporal awareness methods from current literature, and develop innovative approaches tailored to address the identified shortcomings. The efficacy of these methodologies will be evaluated through rigorous testing against the enhanced dataset. \\

Key project objectives include the development of a refined understanding of the temporal limitations present in current LLMs, and the creation of a framework that supports continuous improvement and adaptation of temporal awareness capabilities in IR systems. Through a comprehensive methodology encompassing incident tracking, experimental validation, and performance analysis, this project aims to significantly advance the state-of-the-art in LLM temporal comprehension. \\

Ultimately, this research not only seeks to improve the functional accuracy of ChatUQ but also aims to set a benchmark for future developments in LLM technologies, particularly for applications requiring acute temporal precision. This proposal outlines a detailed plan for addressing these challenges through innovation in methodology, dataset creation, and systematic evaluation.

\thispagestyle{empty} % Suppress page number for abstract
\newpage